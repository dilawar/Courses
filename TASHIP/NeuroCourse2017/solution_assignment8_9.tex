%%=====================================================================================
%%
%%       Filename:  solution_assignment8_9.tex
%%
%%    Description:  Solution to assignment 8 and 9.
%%
%%        Version:  1.0
%%        Created:  04/10/2017
%%       Revision:  none
%%
%%         Author:  Dilawar Singh (), dilawars@ncbs.res.in
%%   Organization:  NCBS Bangalore
%%      Copyright:  Copyright (c) 2017, Dilawar Singh
%%
%%          Notes:  
%%                
%%=====================================================================================

\documentclass[a4paper,10pt]{article}
\usepackage{pgf,tikz,pgfplots}
\usepackage{amsmath}
\usepackage{amssymb}
\usetikzlibrary{shapes,backgrounds,decorations,decorations.pathmorphing}
% Title Page
\title{Title here} 
\author{Dilawar Singh}
\date{\today}
\begin{document}
\maketitle

\paragraph{Problem 8.1, 5 Points}

Supposing you had just two opsins in your retina, for red and blue, that drive
activity in red-selective and blue-selective neurons. Recall that the wavelength
curve for these is quite broad, and let us say that the curves overlap.  Draw a
graph with response on the y axis, and stimulus wavelength on the x axis, for
the two neurons.

\paragraph{Solution} The response curve of these two neurons to the light is
shown below. These are well approximated by following functions $blue(x) =
\exp\left( \frac{-(x-450)^2}{500}\right)$, and $red(x) = \exp\left(
\frac{-(x-550)^2}{2000}\right)$. Notice that these functions are also used to
approximate Dirac delta function.


\begin{tikzpicture}[scale=1]
    \begin{axis}[
            xlabel=Wavelength (nm),ylabel=Response
        , grid style={draw=gray!20}, grid = both
        , no markers
        , xmin = 400, xmax = 700
    ]
    \addplot gnuplot [ raw gnuplot ] {
            set xrange [400:800];
            blue(x)=exp( 20e-4 * -(x-450)^2 );
            plot blue(x) with lines;
    };
    \addplot gnuplot [ raw gnuplot ] {
            set xrange [400:800];
            red(x)=exp( 5e-4 * -(x-550)^2 );
            plot red(x) with lines;
    };
    \end{axis}
\end{tikzpicture}

\paragraph{Problem 8.2, 3 Points}
Consider the response of the red and blue neurons to a monchromatic yellow
stimulus, that has its wavelength 75\% closer to red than to blue. Come up with a
two-colour stimulus that would fool you into thinking it was the same colour as
yellow.

\paragraph{Solution} With values in previous solution, the wavelength is 525 nm.
At this wavelength the overall response is $blue(525) = 0.000013$ and $red(525)
= 0.7316$. The combination of both is roughly 0.7316. The problem is to find
$x$ and $y$ such that $blue(x)+red(x) + blue(y) + red(y) = 0.7316$. 
There are many solutions to this equation. One such solution is $x = 425$ and
$y=590$.

\paragraph{Problem 8.3, 5 Points}
Consider how inputs to a neuron sum, when they are located on an excitable
dendrite. On the x axis draw the number of converging active inputs, and on the
y axis draw the somatic excitation. Consider first summation if the inputs are
far below dendritic action-potential threshold (relatively few inputs), and then
draw what happens to somatic excitation as the number of inputs increases.

\paragraph{Solution} 

\begin{tikzpicture}[scale=1]
    \begin{axis}[
    xlabel=Converging active input,ylabel=Somatic Excitation
    , grid style={draw=gray!20}, grid = both
    ]
    \addplot [color=blue] gnuplot [ raw gnuplot ] {
    };
    \end{axis}
\end{tikzpicture}

\paragraph{Problem 8.4, 5 Points}
Design receptors and thresholds, for a set of 5 single neurons that can  do the
logical operations AND, OR, NOR, NAND and XOR, respectively. Assume only two
inputs in each case, A, and B. Explain how the last two work.

\end{document}          

