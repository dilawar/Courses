%%=====================================================================================
%%
%%       Filename:  solution_5q2q3_6q2.tex
%%
%%    Description:  Solution to problem 5-Q2/Q3 and 6-Q3
%%
%%        Version:  1.0
%%        Created:  03/19/2017
%%       Revision:  none
%%
%%         Author:  Dilawar Singh (), dilawars@ncbs.res.in
%%   Organization:  NCBS Bangalore
%%      Copyright:  Copyright (c) 2017, Dilawar Singh
%%
%%          Notes:  
%%                
%%=====================================================================================

\documentclass[a4paper,10pt]{article}
\usepackage{pgf,tikz}
\usepackage{amsmath}
\usepackage{amssymb}
\usepackage{siunitx}
\usepackage{pgfplots}
\usetikzlibrary{shapes,backgrounds,decorations,decorations.pathmorphing}
% Title Page
\title{Problem 5.2, 5.3 and 6.3} 
\author{Dilawar Singh}
\date{\today}
\begin{document}
\maketitle

\section*{Problem 5.2, 8 points}
\label{par:Problem 5.2}

A cell with electrotonic length L = 2 is voltage-clamped at the soma, to resting
potential = \SI{-65}{\milli \volt}. A synaptic input comes to a distal (far away)
dendrite, onto a synapse with AMPA channels with a tau of \SI{1}{\milli \sec} and a reversal
potential of \SI{0}{\milli\volt}

\begin{enumerate}
    \item Draw the voltage trace at the site of the synapse
    \item Draw the voltage trace halfway to the soma
    \item Draw the voltage trace at the soma
    \item Draw the current recorded by the voltage-clamp.
\end{enumerate}

\emph{I am not very confident about the solution below. Do send me a correct one
if you feel this is wrong.}

\paragraph{Solution} The excitatory post synaptic pulse caused by few channels
is shown below. Its amplitude will increase with number of channels opened by
synaptic input. As long as your amplitude don't go above the \SI{0}{\milli
\volt}, I am OK. Next, to draw waveform halfway to soma, use general passive
propagation principal. For next part, soma is clamped, so you WILL NOT see any
pulse there, soma is fixed at resting potential. Since soma is at resting
potential, voltage clamp has to inject current to keep the soma at resting
potential. Draw that current profile; its time course can be obtained by
voltage seen at soma when there was no voltage-clamped.

\par

\begin{tikzpicture}[scale=1]
    \begin{axis}[
            xlabel=Time (ms),ylabel=mV
        , grid style={draw=gray!20}, grid = both, minor tick num = 4 
        , height = 5cm, width = 10cm
    ]
    \addplot [color=blue] [ raw gnuplot ] gnuplot {
            f(x) =  5 * exp(-x/1) - 65;
            set xrange[-1:4];
            plot x > 0 ? f(x) : -65 ;
    };
    \end{axis}
\end{tikzpicture}

\section*{Problem 5.3, 5 points, Bonus}
A non-myelinated axon (vel = 1 m/s) and a myelinated axon (vel = 100 m/s) both
have the same refractory period of 4 ms. What is the maximum data rate possible
for information transmission along these axons to your toe, 1 metre away? Assume
each action potential carries 2 bits.
    
\paragraph{Solution}

First we need to know how many times we can send an action potential. Let do it
for non-myelinated axon in detail. Time taken for a action potential (signal) to
travel from one end to another is 
$\frac{length}{speed} = \frac{1}{1} = \SI{1}{\sec}$. Since refractory period is
\SI{4}{\milli \sec}, we can send second signal only after than time or in 1
second, we can send maximum of $\frac{1000 ms}{4 ms} = 250$ action potentials
i.e. 500 bits per second. Similarly the rate is \SI{500}{bits \per \sec} for
myelinated neuron as well. The delay in receiving this data depends on speed of
transmission which is \SI{1}{\second} for non-myelinated and \SI{10}{\milli
\second} for the myelinated case.

\section*{Problem 6.2, 5 Points}

The idea of ‘Dendritic Democracy’ states that a neuron will experience the same
depolarization at the cell body, regardless of where the input comes in on a big
dendritic arbor. Knowing what you do now of dendritic signal propagation,
receptors, and HH channels, suggest 3 ways in which the cell could achieve
dendritic democracy. In each case put in some numbers indicating specific
biophysical changes that the cell should undergo to achieve this. Assume the
cell is 2 lambda in electrotonic length.

\paragraph{Solution} Since membrane passive property does not change within a
neuron, the only way to achieve this if a synaptic input causes larger response
at far away site from soma compared to proximal site. This way even though the
signal travels more distance, its influence on soma will be the same.

\begin{itemize}
    \item More receptors at the distal site compared to proximal site.
    \item Properties of HH Channels are different, higher $g_Na$ or lower $g_K$ (due
        to?) at distal site.
    \item The number of vesicle releases at distal site are more or the number
        of transmitters in per vesicle are more. Although its hard to say why
        this should be case at pre-synaptic site.
\end{itemize}

Lets see what you have written!




\end{document}          
