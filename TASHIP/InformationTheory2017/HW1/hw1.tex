% From here  https://raw.githubusercontent.com/jgm/pandoc-templates/master/default.beamer
\documentclass[]{article}
\usepackage{mathtools,amsmath,amssymb,amsthm,mathrsfs}
\usepackage{commath}
\usepackage{lmodern}
\usepackage[sc,osf]{mathpazo}
%\usepackage{circuitikz}
\makeatletter
\@ifpackageloaded{tex4ht}{
    \def\pgfsysdriver{pgfsys-tex4ht.def}
}
\makeatother
\usepackage{pgfplots}
\usepackage{pgfplotstable}
\usepackage[multidot]{grffile}

%% Include newcommands.
\IfFileExists{tex_newcommands.tex}{
    \input{tex_newcommands.tex}
}

\IfFileExists{dot2texi.sty}{
    \usepackage{dot2texi}
}{ }

\usepackage{pgf,tikz}
\usetikzlibrary{shapes,backgrounds,positioning,matrix,decorations}

\providecommand{\tightlist}{%
\setlength{\itemsep}{0pt}\setlength{\parskip}{0pt}}

\usepackage{siunitx}
\usepackage{ifxetex,ifluatex}
\usepackage{listings}
\lstset{ 
    basicstyle=\ttfamily,
    numbers=left,
    numberstyle=\footnotesize,
    stepnumber=2,
    keywordstyle=\color[rgb]{0.13,0.29,0.53}\bfseries,
    stringstyle=\color[rgb]{0.31,0.60,0.02},
    commentstyle=\color[rgb]{0.56,0.35,0.01}\itshape,
    numbersep=5pt,
    backgroundcolor=\color{black!10},
    tabsize=2,
    breaklines=true,
    linewidth=\textwidth
}
% \usepackage[xindyo,acronym,nomain,toc]{glossaries}
% \makeglossaries
%\usepackage[xindy]{imakeidx}
%\makeindex
\setlength{\parskip}{3mm}
\newtheorem{axiom}{Axiom}
\newtheorem{definition}{Definition}
%\newtheorem{comment}{Comment}
\newtheorem{example}{Example}
\newtheorem{lemma}{Lemma}
\newtheorem{corollary}{Corollary}
\newtheorem{property}{Property}
\newtheorem{problem}{Problem}
\newtheorem{remark}{Remark}
\newtheorem{theorem}{Theorem}
\newtheorem{script}{Script}
\usepackage{fixltx2e} % provides \textsubscript
% use upquote if available, for straight quotes in verbatim environments
\IfFileExists{upquote.sty}{\usepackage{upquote}}{}
\ifnum 0\ifxetex 1\fi\ifluatex 1\fi=0 % if pdftex
  \usepackage[utf8]{inputenc}
\else % if luatex or xelatex
  \ifxetex
    \usepackage{mathspec}
    \usepackage{xltxtra,xunicode}
  \else
    \usepackage{fontspec}
  \fi
  \defaultfontfeatures{Mapping=tex-text,Scale=MatchLowercase}
  \newcommand{\euro}{€}
\fi
% use microtype if available
\IfFileExists{microtype.sty}{\usepackage{microtype}}{}
\usepackage[right=5cm, marginparwidth=4cm]{geometry}
\usepackage{longtable,booktabs}
\ifxetex
  \usepackage[setpagesize=false, % page size defined by xetex
              unicode=false, % unicode breaks when used with xetex
              xetex]{hyperref}
\else
  \usepackage[unicode=true]{hyperref}
\fi
\hypersetup{breaklinks=true,
            bookmarks=true,
            pdfauthor={Dilawar Singh},
            pdftitle={Solution to Homework 1},
            colorlinks=true,
            citecolor=blue,
            urlcolor=blue,
            linkcolor=magenta,
            pdfborder={0 0 0}}
\urlstyle{same}  % don't use monospace font for urls
\setlength{\parindent}{0pt}
\setlength{\parskip}{6pt plus 2pt minus 1pt}
\setlength{\emergencystretch}{3em}  % prevent overfull lines
\setcounter{secnumdepth}{5}

\title{Solution to Homework 1}
\author{Dilawar Singh}
\date{\today}
\usepackage{pgfplots}
\usepackage{subfig}
\AtBeginDocument{%
\renewcommand*\figurename{Figure}
\renewcommand*\tablename{Table}
}
\AtBeginDocument{%
\renewcommand*\listfigurename{List of Figures}
\renewcommand*\listtablename{List of Tables}
}
\usepackage{float}
\floatstyle{ruled}
\makeatletter
\@ifundefined{c@chapter}{\newfloat{codelisting}{h}{lop}}{\newfloat{codelisting}{h}{lop}[chapter]}
\makeatother
\floatname{codelisting}{Listing}
\newcommand*\listoflistings{\listof{codelisting}{List of Listings}}

\begin{document}
\maketitle

\section{Coding}\label{coding}

Consider a horse race where the probabilities of winning are given by
negative powers of 2 (1/2, 1/32, etc.). Suppose the lowest possible
probability for any horse is \(2^{-m}\), and that there are \(n_i\)
horses in each probability class \(2^{-i}\).

\begin{enumerate}
\def\labelenumi{\alph{enumi}.}
\tightlist
\item
  What equation must the \(n_i\) satisfy?
\item
  Give 3 possible sets of \(\{n_i\}\) for an 8 horse race (other than
  the uniform case).
\item
  Using the entropy concept, calculate the expected codeword length for
  each set.
\item
  For each set, present an efficient binary coding scheme that achieves
  optimality.
\end{enumerate}

\textbf{Solution}

\textbf{a.} Total probability (probability of all events) must adds up
to 1 i.e. \(\sum n_i * 2^{-i} = 1\)

\textbf{b, c.} One case is trivial that all 8 horses belongs to
probability class \(2^{-3}\). Each horse has the probability of winning
\(\frac{1}{8}\).

Script \texttt{./problem1.hs} generates some solutions (probably all
possible) to this problem. They are listed below.

\begin{longtable}[]{@{}llllll@{}}
\toprule
\(1/2\) & \(1/4\) & \(1/8\) & \(1/16\) & \(1/32\) &
Entropy\tabularnewline
\midrule
\endhead
0 & 0 & 8 & 0 & 0 & 3.0\tabularnewline
0 & 1 & 5 & 2 & 0 & 2.875\tabularnewline
0 & 2 & 2 & 4 & 0 & 2.75\tabularnewline
0 & 2 & 3 & 1 & 2 & 2.6875\tabularnewline
0 & 3 & 0 & 3 & 2 & 2.5625\tabularnewline
0 & 3 & 1 & 0 & 4 & 2.5\tabularnewline
1 & 0 & 1 & 6 & 0 & 2.375\tabularnewline
1 & 0 & 2 & 3 & 2 & 2.3125\tabularnewline
1 & 0 & 3 & 0 & 4 & 2.25\tabularnewline
1 & 1 & 0 & 2 & 4 & 2.125\tabularnewline
\bottomrule
\end{longtable}

Code are not given. Your code must be decodable and its entropy should
be near or equal to calculated entropy given in table above.

\section{{[}CT 2.13{]} Coin weighing.}\label{ct-2.13-coin-weighing.}

Suppose one has n coins, among which there may or may not be one
counterfeit coin. If there is a counterfiet coin, it may be either
heavier or lighter than the other coins. The coins are to be weighed by
a balance.

\begin{enumerate}
\def\labelenumi{\alph{enumi}.}
\tightlist
\item
  Find an upper bound on the number of coins n so that k weighings will
  find the counterfiet coin (if any) and correctly declare it to be
  heavier or lighter.
\item
  Suppose you have k = 3 weighings and 12 coins. How many coins must you
  weigh against each other in the first round?
\item
  (Optional, and hard) Find the full strategy for the 12 coin case.
\end{enumerate}

\textbf{Solution}

Lets not do any guess work. Assume that we draw \(k\) coins and weigh
them with another \(k\) coins.

The probability that counterfiet coins is in selected k coins is
\(p(k)\). It is easy to show that \(p(k) = \frac{k}{12}\). We have three
partitions of size k, k, and 12-k.

We can only weigh partition of size k with other partition of size k.
Weighing k coins with 12-k coins is pointless (proof?). If counterfiet
is in any of k-size partitions, then we reduce the state space to 2k;
otherwise the counterfiet coin is in rest of the 12-k coins (reduce the
state space to 12-k).

With probability \(p(k)\) (that counterfiet coin is in selected k
coins), we reduce the state space by 2k (or to 12-2k); and with
probability \(1-p(k)\) by (12-2k) (or by 2k) i.e. average reduction is
\(w(k)=p(k)(12-2k)+(1-p(k))(2k)\). What value of \(k\) maximize this
function? It is maximum as k=4, and k=5.

\begin{tikzpicture}[scale=1]
    \begin{axis}[ xlabel=k, ylabel=State Space reduction, domain=1:6
        , samples=6
        , title = {State space reduction}
        ]
        \addplot+ [color=blue,] plot { x/12 * (12-2*x) + (1 - x/12) * 2*x };
        \node[pin=-60:{k=4}] at (axis cs:4,6.67) { };
    \end{axis}
\end{tikzpicture}

So to answer \textbf{b}, you can weigh 4 or 5 coins with another equal
number of coins. To answer \textbf{a}, lets assume the genralised case
of N coins, if we weigh k coins with each other,
\(w(k) = 3k - \frac{4k^2}{N}\). Its maximum when \(k = \frac{3N}{8}\).
Let's assume the worst case that the counterfiet coins always end up in
the 2k coins which we weigh against each other. At each weigh, we pick
the optimal number of coins (\(\frac{3N}{8}\)) and reduce the state
space by 2k ( \(\frac{3N}{4}\)). First step we have \(N\) coins, after
one weigh we have \(3N/2\) coins, after second weigh we have
\(\frac{3}{4}\frac{3N}{4}\), and so on. After \(n\) weigh, we have
\(\frac{3}{4}^n N\) coins. We stop when we are left with 2 coins. One
more additional weigh, will tell which one is counterfiet.

\(\frac{3}{4}^n N = 2 \implies n = \frac{\ln{2/N}}{\ln{3/4}} = 3.476 \ln{\frac{N}{2}}\).
Therefore number of maximum weighings are \(3.467\ln{N/2}\). For N=12,
the value is much higher than 3. Needless to say, this is not the most
efficient strategy.

\textbf{Another strategy}

First we divide the coins into three equal parts and weigh one part with
rest of two parts (2 weighing). This will tell us if the counterfiet
coins is heavier or lighter; and also in which N/3 coins it is in.

Once we have done that, we take the N/3 coins, again divide into 3
parts, do one weighing of 1 part with another and we can declare the
part the cointerfeit coin is in. Therefore, starting with N/3, and at
each step dividing the states into 3 parts, after (k-2) weighing, we
must figure out the coin. Therefore
\(\frac{1}{3}^{k-2} \frac{N}{3} = 1 \implies k = 2 + 0.91 \ln \frac{N}{3}\).
For N = 12, we have \(k = 3.26\); very near to optimal solution.

\end{document}
