\documentclass[answers,addpoints]{exam}
\usepackage[margin=10mm]{geometry}
\usepackage{amsthm}
\usepackage{float} %  figure inside minipage 
\usepackage{ifthen,empheq}
\usepackage[]{graphicx}
\usepackage[]{minted}
\usepackage{amssymb}
\usepackage[multidot]{grffile}
\usepackage{pgfplots}
\usepackage{pgfplotstable}
\setlength{\headheight}{20pt}

\usepackage{tikz}
\usepgfplotslibrary{external}
\tikzexternalize[prefix=_tikz/,shell escape=-shell-escape]
\tikzset{external/system call={pdflatex \tikzexternalcheckshellescape -halt-on-error -interaction=batchmode -jobname "\image" "\texsource"}}
\immediate\write18{mkdir -p _tikz}

\title{Solution to HW2}
\author{Dilawar Singh}
\usepackage{hyperref}

\begin{document}
\Large
\maketitle

\begin{questions}

    \question[20]

    Refer to homework statement.

    \begin{solution}

        Except for the last subplot, these plots shows how each function
        will transform given functions in homework. These two functions are in
        first subplot. They are called \texttt{fun1} and \texttt{fun2}.

        \begin{figure}[H]
        \begin{center}
            \includegraphics[width=1\textwidth]{./solution1.png}
        \end{center}
        \caption{Solution to problem 1}
        \label{fig:}
        \end{figure}

    \end{solution}

    \question[12]
    Refer to homework statement.

    \begin{solution}

        See figure \ref{fig:sol2}.

        \begin{figure}[H]
            \begin{center}
                \includegraphics[width=1\textwidth]{solution2.png}
            \end{center}
            \caption{Solution 2}
            \label{fig:sol2}
        \end{figure}

    \end{solution}

    \question[5]

    Show that if $f(x) = \sqrt[3]{x}$, then for any $a \in \mathbb{R} \setminus
    \{0\}\; f'(a) = 1/3a^{-2/3}$.

    \begin{solution}

        We start with $f(a+h) = f(a) + h f'(a) + R(h)$ and $R(h)/h
        \rightarrow 0$ when $h \rightarrow 0$. We can write the following:

        \begin{align}
            f'(a)  = \lim_{h \rightarrow 0} \frac{ f(a+h) - f(a) }{h}  
        \end{align}

        This leads to,

        \begin{align}
            f(\sqrt[3]{x}) &= \lim_{h \rightarrow 0} \frac{ \sqrt[3]{x+h} - \sqrt[3]{x} }{h}  \\
                           &= \lim_{h \rightarrow 0} \frac{ (x+h)^{1/3} - x^{1/3} }{h} \\
                           &= \lim_{h \rightarrow 0} \frac{ x^{1/3} \left( 
        \left( 1 + \frac{h}{x}  \right)^{1/3} -1 \right)  }{h}
                           &= \lim_{h \rightarrow 0} \frac{ x^{1/3} \left( 
        1 + \frac{h}{3x} + \ldots  -1 \right)  }{h} \\
                            &= x^{1/3} \left( \frac{h}{3xh} \right) \\
                            &= x^{-2/3} \left( \frac{1}{3} \right) 
        \end{align}


    \end{solution}

    \question[5]
    Show that if $g(x) = \frac{1}{f(x)}$, then for any $a \in \mathbb{R}$ such
    that $f(a) \neq 0, g'(a) = -\frac{f'(a)}{f(a)^2}$.

    \begin{solution}

        \begin{align}
            g'(a) &= \lim_{h \rightarrow 0} \frac{g(a+h)-g(a)}{h} \\
                  &= \lim_{h \rightarrow 0} \frac{ \frac{1}{f(a+h)}-\frac{1}{f(a)}}{h} \\
                  &= \lim_{h \rightarrow 0} \frac{ f(a) - f(a+h) }{h f(a)
            f(a+h)}  \\
            &= {\bf \lim_{h \rightarrow 0} - \frac{ f(a+h) - f(h) }{h}} \frac{1}{f(a) f(a+h)}  \\
            &= - f'(a) \frac{1}{f(a)f(a)}
        \end{align}
        
    \end{solution}

    \question[5]
    Write the approximation for $\sqrt{99.8}$ and $\sqrt[3]{5.01^2 + 2}$.

    \begin{solution}
        
    \end{solution}

    \question[8]
    Differentiate each of the following $\ldots$.

    \begin{solution}
        
    \end{solution}

    \question[8] { Inverse function theorem }

    \begin{parts}
        \part[4]
        \begin{solution}

            \begin{align}
                f(x) &= \sqrt{ 1 + x^3 }                \\
                \implies f^{-1}(x) &= \frac{1}{\sqrt{1+x^3}}  \\
                (f^{-1})'(x) &= - \frac{1}{2}( 1 + x^3 )^{-3/2} \\
            \end{align}
        \end{solution}

        \part[4]
        \begin{solution}

            \begin{align}
                (f^{-1}(x))' &=  - f^{-2}(x) f'(x) \\
                             &= - (1+x^3)^{-1} \frac{1}{2} (1+x^3) ^ {-1/2} \\
                             &= - \frac{1}{2} (1+x^3)^{-3/2}
            \end{align}
            
        \end{solution}


    \end{parts}

        

\end{questions}

\end{document}          
