\documentclass[]{article}
\usepackage{amsmath,amssymb,amsthm}
\usepackage[utf8]{inputenc}
\usepackage{lmodern}
%\usepackage{circuitikz}
\makeatletter
\@ifpackageloaded{tex4ht}{
    \def\pgfsysdriver{pgfsys-tex4ht.def}
}
\makeatother
\usepackage{pgfplots}
\usepackage{pgfplotstable}
\usepackage{pgf,tikz}
\usetikzlibrary{shapes,backgrounds,positioning,matrix,decorations}

\usepackage{siunitx}
\usepackage{python}
\usepackage{ifxetex,ifluatex}
\usepackage{listings}
\setlength{\parskip}{3mm}
\newtheorem{axiom}{Axiom}
\newtheorem{definition}{Definition}
\newtheorem{comment}{Comment}
\newtheorem{example}{Example}
\newtheorem{lemma}{Lemma}
\newtheorem{property}{Property}
\newtheorem{problem}{Problem}
\newtheorem{remark}{Remark}
\newtheorem{theorem}{Theorem}
\newtheorem{script}{Script}

\usepackage{fixltx2e} % provides \textsubscript
% use upquote if available, for straight quotes in verbatim environments
\IfFileExists{upquote.sty}{\usepackage{upquote}}{}
\ifnum 0\ifxetex 1\fi\ifluatex 1\fi=0 % if pdftex
  \usepackage[utf8]{inputenc}
\else % if luatex or xelatex
  \ifxetex
    \usepackage{mathspec}
    \usepackage{xltxtra,xunicode}
  \else
    \usepackage{fontspec}
  \fi
  \defaultfontfeatures{Mapping=tex-text,Scale=MatchLowercase}
  \newcommand{\euro}{€}
\fi
% use microtype if available
\IfFileExists{microtype.sty}{\usepackage{microtype}}{}
\ifxetex
  \usepackage[setpagesize=false, % page size defined by xetex
              unicode=false, % unicode breaks when used with xetex
              xetex]{hyperref}
\else
  \usepackage[unicode=true]{hyperref}
\fi
\hypersetup{breaklinks=true,
            bookmarks=true,
            pdfauthor={Dilawar Singh},
            pdftitle={Assignment 1},
            colorlinks=true,
            citecolor=blue,
            urlcolor=blue,
            linkcolor=magenta,
            pdfborder={0 0 0}}
\urlstyle{same}  % don't use monospace font for urls
\setlength{\parindent}{0pt}
\setlength{\parskip}{6pt plus 2pt minus 1pt}
\setlength{\emergencystretch}{3em}  % prevent overfull lines
\setcounter{secnumdepth}{0}

\title{Assignment 1}
\author{Dilawar Singh}
\date{}

\begin{document}
\maketitle

\newcommand{\mean}[1]{\left< #1 \right>}

\section{Problem 1}\label{problem-1}

The mean and variance of a random variable x are defined as
$\mu = \mean{x}, \sigma_x^2 = \mean{x - \mean{x}} $ where the triangular
brackets represent population averages. From these definitions, derive a
formula for the variance purely in terms of the first two
\textbf{moments} of the distribution:
$\sigma_x^2 = f(\mean{x}, \mean{x^2}$.

\subsection{Solution}\label{solution}

\[ 
\begin{aligned}
\sigma_x^2 &= \mean{(x - \mean{x})^2}  \\
    &= \mean{x^2 + \mean{x}^2 - 2 x \mean{x}}  \\
    &= \mean{x^2} + \mean{\mean{x}^2} - \mean{2x\mean{x}} \\
    &= \mean{x^2} + \mean{x}^2 - 2\mean{x}\mean{x} \\
    &= \mean{x^2} + \mean{x}^2 - 2\mean{x}^2  \\
    &= \mean{x^2} - \mean{x}^2 
\end{aligned}
\]

\section{Problem 2}\label{problem-2}

In class we discussed the 1-dimensional random walk with stepsize
$\delta$ , timestep $\tau$, and equal probability of going left or
right. After N steps, the chance of a net displacement $k\delta$ is
given by the probability that the walker took k more steps to the right
than to the left:

\[N_{left}=(N-k)/2,N_{right}=(N+k)/2,N_{right}-N_{left}=k\]

Using the binomial theorem, the chance of this happening is

\[ p(x=k\delta)=\frac{N!}{\frac{N-k}{2}!\frac{N+k}{2}!}\frac{1}{2^N}\].

By Stirling's approximation, $\ln(N!) \sim N\ln N-N$. Apply this, and
Taylor expand upto square terms in $k/N$ (since we expect the maximum to
be at $k = 0$, and $k/N << 1$), to derive the following approximate form
for this distribution: \[ p(x = k \delta) \sim  e^{-\frac{k^2}{2N}} \]

The normalized form of this is given in the Ramaswamy paper.

\subsection{Solution}\label{solution-1}

\end{document}
